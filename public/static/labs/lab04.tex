% Options for packages loaded elsewhere
\PassOptionsToPackage{unicode}{hyperref}
\PassOptionsToPackage{hyphens}{url}
%
\documentclass[
]{article}
\usepackage{amsmath,amssymb}
\usepackage{lmodern}
\usepackage{iftex}
\ifPDFTeX
  \usepackage[T1]{fontenc}
  \usepackage[utf8]{inputenc}
  \usepackage{textcomp} % provide euro and other symbols
\else % if luatex or xetex
  \usepackage{unicode-math}
  \defaultfontfeatures{Scale=MatchLowercase}
  \defaultfontfeatures[\rmfamily]{Ligatures=TeX,Scale=1}
\fi
% Use upquote if available, for straight quotes in verbatim environments
\IfFileExists{upquote.sty}{\usepackage{upquote}}{}
\IfFileExists{microtype.sty}{% use microtype if available
  \usepackage[]{microtype}
  \UseMicrotypeSet[protrusion]{basicmath} % disable protrusion for tt fonts
}{}
\makeatletter
\@ifundefined{KOMAClassName}{% if non-KOMA class
  \IfFileExists{parskip.sty}{%
    \usepackage{parskip}
  }{% else
    \setlength{\parindent}{0pt}
    \setlength{\parskip}{6pt plus 2pt minus 1pt}}
}{% if KOMA class
  \KOMAoptions{parskip=half}}
\makeatother
\usepackage{xcolor}
\IfFileExists{xurl.sty}{\usepackage{xurl}}{} % add URL line breaks if available
\IfFileExists{bookmark.sty}{\usepackage{bookmark}}{\usepackage{hyperref}}
\hypersetup{
  pdftitle={Lab 04:},
  hidelinks,
  pdfcreator={LaTeX via pandoc}}
\urlstyle{same} % disable monospaced font for URLs
\usepackage[margin=1in]{geometry}
\usepackage{color}
\usepackage{fancyvrb}
\newcommand{\VerbBar}{|}
\newcommand{\VERB}{\Verb[commandchars=\\\{\}]}
\DefineVerbatimEnvironment{Highlighting}{Verbatim}{commandchars=\\\{\}}
% Add ',fontsize=\small' for more characters per line
\usepackage{framed}
\definecolor{shadecolor}{RGB}{248,248,248}
\newenvironment{Shaded}{\begin{snugshade}}{\end{snugshade}}
\newcommand{\AlertTok}[1]{\textcolor[rgb]{0.94,0.16,0.16}{#1}}
\newcommand{\AnnotationTok}[1]{\textcolor[rgb]{0.56,0.35,0.01}{\textbf{\textit{#1}}}}
\newcommand{\AttributeTok}[1]{\textcolor[rgb]{0.77,0.63,0.00}{#1}}
\newcommand{\BaseNTok}[1]{\textcolor[rgb]{0.00,0.00,0.81}{#1}}
\newcommand{\BuiltInTok}[1]{#1}
\newcommand{\CharTok}[1]{\textcolor[rgb]{0.31,0.60,0.02}{#1}}
\newcommand{\CommentTok}[1]{\textcolor[rgb]{0.56,0.35,0.01}{\textit{#1}}}
\newcommand{\CommentVarTok}[1]{\textcolor[rgb]{0.56,0.35,0.01}{\textbf{\textit{#1}}}}
\newcommand{\ConstantTok}[1]{\textcolor[rgb]{0.00,0.00,0.00}{#1}}
\newcommand{\ControlFlowTok}[1]{\textcolor[rgb]{0.13,0.29,0.53}{\textbf{#1}}}
\newcommand{\DataTypeTok}[1]{\textcolor[rgb]{0.13,0.29,0.53}{#1}}
\newcommand{\DecValTok}[1]{\textcolor[rgb]{0.00,0.00,0.81}{#1}}
\newcommand{\DocumentationTok}[1]{\textcolor[rgb]{0.56,0.35,0.01}{\textbf{\textit{#1}}}}
\newcommand{\ErrorTok}[1]{\textcolor[rgb]{0.64,0.00,0.00}{\textbf{#1}}}
\newcommand{\ExtensionTok}[1]{#1}
\newcommand{\FloatTok}[1]{\textcolor[rgb]{0.00,0.00,0.81}{#1}}
\newcommand{\FunctionTok}[1]{\textcolor[rgb]{0.00,0.00,0.00}{#1}}
\newcommand{\ImportTok}[1]{#1}
\newcommand{\InformationTok}[1]{\textcolor[rgb]{0.56,0.35,0.01}{\textbf{\textit{#1}}}}
\newcommand{\KeywordTok}[1]{\textcolor[rgb]{0.13,0.29,0.53}{\textbf{#1}}}
\newcommand{\NormalTok}[1]{#1}
\newcommand{\OperatorTok}[1]{\textcolor[rgb]{0.81,0.36,0.00}{\textbf{#1}}}
\newcommand{\OtherTok}[1]{\textcolor[rgb]{0.56,0.35,0.01}{#1}}
\newcommand{\PreprocessorTok}[1]{\textcolor[rgb]{0.56,0.35,0.01}{\textit{#1}}}
\newcommand{\RegionMarkerTok}[1]{#1}
\newcommand{\SpecialCharTok}[1]{\textcolor[rgb]{0.00,0.00,0.00}{#1}}
\newcommand{\SpecialStringTok}[1]{\textcolor[rgb]{0.31,0.60,0.02}{#1}}
\newcommand{\StringTok}[1]{\textcolor[rgb]{0.31,0.60,0.02}{#1}}
\newcommand{\VariableTok}[1]{\textcolor[rgb]{0.00,0.00,0.00}{#1}}
\newcommand{\VerbatimStringTok}[1]{\textcolor[rgb]{0.31,0.60,0.02}{#1}}
\newcommand{\WarningTok}[1]{\textcolor[rgb]{0.56,0.35,0.01}{\textbf{\textit{#1}}}}
\usepackage{graphicx}
\makeatletter
\def\maxwidth{\ifdim\Gin@nat@width>\linewidth\linewidth\else\Gin@nat@width\fi}
\def\maxheight{\ifdim\Gin@nat@height>\textheight\textheight\else\Gin@nat@height\fi}
\makeatother
% Scale images if necessary, so that they will not overflow the page
% margins by default, and it is still possible to overwrite the defaults
% using explicit options in \includegraphics[width, height, ...]{}
\setkeys{Gin}{width=\maxwidth,height=\maxheight,keepaspectratio}
% Set default figure placement to htbp
\makeatletter
\def\fps@figure{htbp}
\makeatother
\setlength{\emergencystretch}{3em} % prevent overfull lines
\providecommand{\tightlist}{%
  \setlength{\itemsep}{0pt}\setlength{\parskip}{0pt}}
\setcounter{secnumdepth}{-\maxdimen} % remove section numbering
\ifLuaTeX
  \usepackage{selnolig}  % disable illegal ligatures
\fi

\title{Lab 04:}
\author{}
\date{\vspace{-2.5em}Due Monday July 18 at 11:59pm}

\begin{document}
\maketitle

By the end of this lab you will

\begin{itemize}
\tightlist
\item
  compute probabilities
\item
  build a logistic regression model and assess its performance
\end{itemize}

\hypertarget{getting-started}{%
\subsection{Getting started}\label{getting-started}}

~1. Download the lab template by pasting the code below in your
\textbf{console}:

\begin{verbatim}
download.file("https://sta101.github.io/static/labs/lab04_template.Rmd",
              destfile = "lab04.rmd")
\end{verbatim}

~2. Under the ``Files'' tab on the right hand side, click on
\texttt{lab04.rmd} to open the lab template.

~3. Complete the exercises below using the space provided.

\hypertarget{warm-up}{%
\subsection{Warm up}\label{warm-up}}

Be sure to update the YAML at the top of the document to include your
name and the date.

\hypertarget{packages}{%
\subsubsection{Packages}\label{packages}}

\begin{Shaded}
\begin{Highlighting}[]
\FunctionTok{library}\NormalTok{(tidyverse)}
\FunctionTok{library}\NormalTok{(tidymodels)}
\FunctionTok{library}\NormalTok{(knitr)}
\end{Highlighting}
\end{Shaded}

\hypertarget{data}{%
\subsubsection{Data}\label{data}}

Load the data:

\begin{Shaded}
\begin{Highlighting}[]
\NormalTok{parkinsons }\OtherTok{=} \FunctionTok{read\_csv}\NormalTok{(}\StringTok{"https://sta101.github.io/static/labs/data/parkinsons\_cleaned.csv"}\NormalTok{)}
\end{Highlighting}
\end{Shaded}

This dataset comes from
\href{https://www.ncbi.nlm.nih.gov/pmc/articles/PMC3051371/}{Little et
al.~(2008)}. The data includes various measurements of dysphonia from 32
people, 24 with Parkinson's disease (PD). Multiple measurements were
taken per individual. The measurements we examine in this subset of the
data include:

\begin{itemize}
\tightlist
\item
  \texttt{name}: patient ID
\item
  \texttt{jitter}: a measure of relative variation in fundamental
  frequency
\item
  \texttt{shimmer}: a measure of variation in amplitude (dB)
\item
  \texttt{PPE}: pitch period entropy
\item
  \texttt{HNR}: a ratio of total components vs.~noise in the voice
  recording
\item
  \texttt{status}: health status (1 for PD, 0 for healthy)
\end{itemize}

\hypertarget{exercises}{%
\subsection{Exercises}\label{exercises}}

\begin{enumerate}
\def\labelenumi{\arabic{enumi}.}
\item
  What are the identification codes (\texttt{names}) of healthy
  individuals in the data set? Print your output as a nice
  \texttt{kable} table.
\item
  What is the probability a randomly selected \textbf{observation} from
  this data set is an observation of an individual with Parkinson's
  disease? What is the probability a randomly selected
  \textbf{participant} (different than an observation) in this data set
  has PD?
\item
  Create a new column that classifies vocal amplitude variability as
  high (\textgreater=.30 dB) or low (\textless30). Is high vocal
  amplitude independent of whether or not someone has PD? Why or why
  not?
\item
  What is the probability \texttt{HNR} is greater than 25 given that the
  participant is in the control group?
\item
  Next you will build a predictive model of PD status, but first split
  the data into two disjoint sets: a training set and a test set. From
  the original data frame, remove 4 PD and 4 healthy individuals to be
  in your test set. For consistency, choose the 4 healthy individuals
  with the lowest ID number of their respective category,
  e.g.~\texttt{phon\_R01\_S07} ends with \texttt{07} (the lowest number
  of the healthy group) so they should be placed in the test data frame.
  Similarly, the lowest ID number for an individual with PD is
  \texttt{S01} so \texttt{phon\_R01\_S01} should also be placed in the
  test data frame.
\end{enumerate}

Your train data frame should contain 147 rows and your test data frame
should contain 48. In your code chunk, print the number of rows of each
data frame.

Big hint: create a group of individuals to be in your test data frame,
e.g.~\texttt{keep\_ids\ =\ c("id1",\ "id2",\ "id3")} and pair this with
filter logic described in \texttt{ae3}.

\begin{enumerate}
\def\labelenumi{\arabic{enumi}.}
\setcounter{enumi}{5}
\item
  Fit a main effects logistic regression model that predicts prob(PD)
  status from \texttt{HNR}, \texttt{PPE}, \texttt{jitter} and
  \texttt{shimmer}. Print your model \texttt{tidy}.
\item
  Edit the code chunk below, specifically renaming \texttt{model\_fit}
  and \texttt{test\_data} where appropriate. Un-comment and run to find
  the predicted probabilities of Parkinson's disease in the test data
  frame.
\end{enumerate}

\begin{Shaded}
\begin{Highlighting}[]
\CommentTok{\# prediction = predict(model\_fit, test\_data, type = "prob")}

\CommentTok{\# test\_result = test\_data \%\textgreater{}\%}
\CommentTok{\#   mutate(predicted\_prob\_pd = prediction$.pred\_1)}
\end{Highlighting}
\end{Shaded}

Next, create a new column that classifies an individual as having PD if
the predicted probability is above 50\%. Repeat with a decision boundary
of 75\% and 90\%.

How many false positives do you have in each case? False negatives? If
you were to use your model as a diagnostic tool for PD to decide if
someone should undergo subsequent testing, which decision boundary would
you prefer and why?

Note: your narrative should read, e.g.: ``With a decision boundary of
50\%, my model yields X false positives and Y false negatives. With a
decision boundary of 75\%\ldots{}'' etc.

\hypertarget{formatting}{%
\subsection{Formatting}\label{formatting}}

\textbf{Reminder}: For all assignments in this course there is a
``formatting'' component to the grade. To receive full points for
``formatting'', you must:

~1. Have your name at the top of the knitted document

~2. Pipes \texttt{\%\textgreater{}\%} and ggplot layers \texttt{+}
should be followed by a newline (see formatting above)

~3. Your code should be under the 80 character code limit. (You
shouldn't have to scroll to see all your code on the knitted document).

~4. All exercises and corresponding pages should be linked on
gradescope.

These necessary ``tidyverse'' style choices are good general practice
and will help make your code more legible. For a more extensive list of
recommended guidelines, \href{https://style.tidyverse.org/}{click here}.

\hypertarget{submitting-to-gradescope}{%
\subsection{Submitting to gradescope}\label{submitting-to-gradescope}}

Once you are fully satisfied with your lab, Knit to .pdf to create a
.pdf document. You may notice that the formatting/theme of the report
has changed -- this is expected. Remember -- you must turn in a .pdf
file to the Gradescope page before the submission deadline for credit.
To submit your assignment:

\begin{itemize}
\item
  Go to \url{http://www.gradescope.com} and click Log in in the top
  right corner. - Click \texttt{School\ Credentials},
  \texttt{Duke\ NetID} and log in using your NetID credentials.
\item
  Click on your STA 101 course.
\item
  Click on the assignment, and you'll be prompted to submit it.
\item
  Mark the pages associated with each exercise. All of the papers of
  your lab should be associated with at least one question (i.e., should
  be ``checked''). Select the first page of your .pdf submission to be
  associated with the ``Formatting'' section.
\end{itemize}

\hypertarget{grading}{%
\subsection{Grading}\label{grading}}

Total: 50 pts.

\begin{verbatim}
Exercise 1: 3pts

Exercise 2: 6pts

Exercise 3: 4pts

Exercise 4: 4pts

Exercise 5: 4pts

Exercise 6: 4pts

Exercise 7: 6pts

Exercise 8: 5pts

Exercise 9: 6pts

Exercise 10: 3pts

Exercise 11: 3pts

Workflow and formatting:  2pts
\end{verbatim}

\end{document}
